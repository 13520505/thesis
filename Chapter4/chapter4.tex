\chapter{THỰC NGHIỆM VÀ ĐÁNH GIÁ}
\ifpdf
    \graphicspath{{Chapter4/Chapter4Figs/PNG/}{Chapter4/Chapter4Figs/PDF/}{Chapter4/Chapter4Figs/}}
\else
    \graphicspath{{Chapter4/Chapter4Figs/EPS/}{Chapter4/Chapter4Figs/}}
\fi

\section{Mở đầu}
Mục đích của chương này là trình bày một số kết quả thực nghiệm trên bộ dữ liệu thu thập được. Qua đó đánh giá, nhận định và so sánh các thuật toán phân lớp.

\section{Tổng quan về bộ dữ liệu}
Bộ dữ liệu gồm các bài viết từ nhiều nguồn tin tức. Dữ liệu được lấy thông qua luồng: Crawler . Một danh sách hơn 700 nguồn được sử dụng để lấy dữ liệu thông tin từ trên mạng.

Mỗi tin trong tập dữ liệu bao gồm các thông tin sau: Id , title của tin tức, nội dung tin tức, nguồn đăng tin tức, thời gian tin tức được post, thời gian lấy về cơ sở dữ liệu MySQL, thời gian tin tức được thu thập vào hệ thống, description của bài báo, đường dẫn link của tin tức, tác giả.

Bộ dữ liệu thử nghiệm gần nhất gồm 15,612 tin tiếng Việt, được thu thập thông qua Crawler trong khoảng thời gian 11 ngày, từ 31/1/2017 đến 10/2/2017.

Dữ liệu thống kê cho tập dữ liệu sử dụng để train các model:
	\begin{table}[H]
		\centering
		\setlength\extrarowheight{3pt}
		\begin{tabular}{|l|l|l|}
			\hline
			Dữ liệu gán nhãn & Số lượng \\
			\hline
			Không rác   & 7331\\
			\hline
			Rác   & 8281\\
			\hline
			\hline
			Tổng   & 15612\\
			\hline
		\end{tabular}%
		\caption{Thống kê dữ liệu gán nhãn} \label{tab:table_4_1}%
	\end{table}
	\begin{table}[H]
		\centering
		\setlength\extrarowheight{3pt}
		\begin{tabular}{|l|l|l|}
			\hline
			Dữ liệu gán nhãn & Số lượng \\
			\hline
			Quảng cáo   & 4627\\
			\hline
			Chia sẻ   & 3279\\
			\hline
			Tuyển dụng   & 375\\
			\hline
			\hline
			Tổng   & 8281\\
			\hline
		\end{tabular}%
		\caption{Thống kê loại rác} \label{tab:table_4_2}%
	\end{table}
	\begin{table}[H]
%		\centering
		\setlength\extrarowheight{3pt}
		\begin{tabular}{|p{4cm}|p{10cm}|}
			\hline
			Accident      & tai nạn, chết người, thảm khốc, rơi máy bay, tai nạn máy bay, máy bay mất tích, \#tainan, \#tainạn, \#chetnguoi, \#chếtngười \\
			\hline
			Act of War or Violence, Military News & thả bom, đánh bom, tên lửa, hạt nhân, bom hạt nhân, đầu đạn hạt nhân, vũ khí, vũ khí huỷ diệt, xả súng, khủng bố, \#đánhbom, \#khủngbố \\
			\hline
			Bizarre News and World Records & kỷ lục thế giới, Guinness, chuyện lạ có thật, chuyện lạ khó tin, chuyện lạ bốn phương, phong tục kỳ lạ \\
			\hline
			Celebrity and Human Interest News & tổng thống, tổng bí thư, thủ tướng, chủ tịch nước, phát ngôn, phát ngôn gây sốc, Barack Obama, Donald Trump, Putin, Shinzo Abe, Tập Cận Bình, Rodrigo Duterte, Nguyễn Phú Trọng, Nguyễn Xuân Phúc, Trần Đại Quang, Ca sĩ, Hoa hậu, Người mẫu,	Diễn viên \\
			\hline
			Election      & đại hội đảng, bầu cử, bầu cử đại biểu quốc hội, bầu cử tổng thống, \#baucu, \#bầucử \\
			\hline
			Financial     & xăng lên giá, xăng giảm giá, chứng khoán, bất động sản, chấn động thị trường, \#taichinh, \#tàichính \\
			\hline
			General       & tin nóng, tin giật gân, \#tinnong, \#tinnóng, \#tingiatgan, \#tingiậtgân \\
			\hline
			Legal and Criminal Case & cướp, giết người, khủng khiếp, hãm hiếp, hiếp dâm, lạm dụng tình dục, tham ô, tham nhũng, \#gietnguoi, \#giếtngười \\
			\hline
			Natural Disaster & động đất, sóng thần, lũ lụt, mưa đá, thiên tai, bão lớn, \#thientai, \#thiêntai \\
			\hline
			New Law       & dự thảo luật, điều luật mới, chính sách mới \\
			\hline
			Political Meeting and Statement & tổng thống, tổng bí thư, thủ tướng, chủ tịch nước, hội nghị, cuộc gặp gỡ, gặp mặt, họp mặt \\
			\hline
			Scandal and Hearing & bê bối, quấy rối, quấy rối tình dục, hầu tòa, ra tòa, scandal, \#scandal \\
			\hline
			Science and Discovery & phát minh, giải nobel, giải fields, khám phá mới, phát hiện mới \\
			\hline
			Sport         & đội tuyển bóng đá quốc gia, đội tuyển U19, U19 Hoàng Anh Gia Lai, SeaGames, AFC Suzuki Cup, \#bongda, \#bóngđá \\
			\hline
		\end{tabular}%
		\caption{Danh sách từ khóa để thu thập dữ liệu}
		\label{tab:twitterkeywords}%
	\end{table}%

Bộ dữ liệu có thể được tải về từ địa chỉ:
\url{https://drive.google.com/open?id=1xZVBcaVtZAmQ4xU0KKPvB5ZRAUoKc2_v}

\section{Thiết lập thực nghiệm, cách đánh giá}
Sử dụng 3 thuật toán phân lớp: thuật toán Naive Bayes, thuật toán J48, thuật toán Support Vector Machine. 

Ta đánh giá và so sánh kết quả về thời gian xử lý và chất lượng phân lớp thông qua một số độ đo đã trình bày ở mục \ref{clusterEvalMetrics}: Precision, Recall, F-Measure, ROC Area, confusion matrix.

%Cả 4 thuật toán đều sử dụng giá trị ngưỡng để xét điểm dữ liệu có thuộc cụm hay không. Thuật toán Nearest Neighbor và Boost Named Entity sử dụng ngưỡng về \textit{độ tương đồng} MergeThreshold, ngược lại thuật toán LSH sử dụng ngưỡng \textit{khoảng cách} NoveltyThreshold. Thực chất $	NoveltyThreshold = 1 - MergeThreshold$ nên ta vẫn có thể quy đổi giá trị giữa chúng để so sánh kết quả các thuật toán. Tất cả giá trị ngưỡng trong chương này sẽ được quy đổi về giá trị của \textbf{MergeThreshold} tương ứng.

%Riêng thuật toán LSH có thêm các thông số như: số hashtable, số siêu phẳng, lượng document tối đa cho mỗi bucket,... ta sẽ cần xét sự ảnh hưởng của chúng đến quá trình và kết quả gom cụm.
\section{Kết quả thực nghiệm}

	\subsection{Kết quả train model phân loại tin tức}
	Dưới đây là kết quả các thuật toán và kết quả của một số độ đo :
	\subsubsection{Kết quả dựa trên nội dung của tin của tập mẫu}
	\begin{table}[H]
		\centering
		\setlength\extrarowheight{3pt}
		\begin{tabular}{|l|l|l|}
			\hline
			Dữ liệu gán nhãn & Số lượng \\
			\hline
			Mẫu dữ liệu   & 15612\\
			\hline
			Số chiều   & 111363\\
			\hline
			Cross-validation:   & 10\\
			\hline
		\end{tabular}%
		\caption{Thông số cơ bản của tập train} \label{tab:table_4_3}%
	\end{table}
	\begin{table}[H]
		\centering
		\begin{tabular}{|l|l|l|}
			\hline
			Time build model(seconds): &            &          \\
			\hline
			SVM                        & NaiveBayes & J48      \\
			\hline
			382.85                     & 853.67     & 43530.21 \\
			\hline
		\end{tabular}
		\caption{Thời gian train model}
		\label{tab:table_4_4}
	\end{table}
	\begin{table}[H]
	\centering
	\rotatebox{90}{
		\begin{tabular}{|c|c|c|c|c|c|c|c|c|c|c|c|c|}
			\hline
			\multirow{2}{*}{Class} & \multicolumn{3}{c|}{Precision} & \multicolumn{3}{c|}{Recall} & \multicolumn{3}{c|}{F-Measure} & \multicolumn{3}{c|}{ROC Area} \\ \cline{2-13} 
			& J48     & SVM    & NaiveBayes  & J48    & SVM   & NaiveBayes & J48     & SVM    & NaiveBayes  & J48    & SVM    & NaiveBayes  \\ \hline
			Rác                    & 0.845   & 0.876  & 0.847       & 0.838  & 0.915 & 0.874      & 0.842   & 0.895  & 0.860       & 0.823  & 0.884  & 0.854       \\ \hline
			Tin                    & 0.819   & 0.898  & 0.852       & 0.827  & 0.854 & 0.821      & 0.823   & 0.876  & 0.836       & 0.823  & 0.884  & 0.871       \\ \hline
			& 0.832   & 0.887  & 0.849       & 0.833  & 0.886 & 0.849      & 0.832   & 0.886  & 0.849       & 0.823  & 0.884  & 0.862       \\ \hline
		\end{tabular}
		}
	\caption{Kết quả train model dựa trên một số độ đo}
	\label{tab:table_4_5}
	\end{table}%
	\subsubsection{Kết quả dựa trên tiêu đề của tin của tập mẫu}
	Dưới đây là kết quả các thuật toán và kết quả của một số độ đo :
	\begin{table}[H]
		\centering
		\setlength\extrarowheight{3pt}
		\begin{tabular}{|l|l|l|}
			\hline
			Dữ liệu gán nhãn & Số lượng \\
			\hline
			Mẫu dữ liệu   & 15612\\
			\hline
			Số chiều   & 15133\\
			\hline
			Cross-validation:   & 10\\
			\hline
		\end{tabular}%
		\caption{Thông số cơ bản của tập train} \label{tab:table_4_6}%
	\end{table}
	\begin{table}[H]
		\centering
		\begin{tabular}{|l|l|l|}
			\hline
			Time build model(seconds): &            &          \\
			\hline
			SVM                        & NaiveBayes & J48      \\
			\hline
			66.23                     & 130.78     & 5895.41	\\
			\hline
		\end{tabular}
		\caption{Thời gian train model}
		\label{tab:table_4_7}
	\end{table}
	\begin{table}[H]
		\centering
		\rotatebox{90}{
			\begin{tabular}{|c|c|c|c|c|c|c|c|c|c|c|c|c|}
				\hline
				\multirow{2}{*}{Class} & \multicolumn{3}{c|}{Precision} & \multicolumn{3}{c|}{Recall} & \multicolumn{3}{c|}{F-Measure} & \multicolumn{3}{c|}{ROC Area} \\ \cline{2-13} 
				& J48     & SVM    & NaiveBayes  & J48    & SVM   & NaiveBayes & J48     & SVM    & NaiveBayes  & J48    & SVM    & NaiveBayes  \\ \hline
				Rác                    & 0.845   & 0.876  & 0.847       & 0.838  & 0.915 & 0.874      & 0.842   & 0.895  & 0.860       & 0.823  & 0.884  & 0.854       \\ \hline
				Tin                    & 0.819   & 0.898  & 0.852       & 0.827  & 0.854 & 0.821      & 0.823   & 0.876  & 0.836       & 0.823  & 0.884  & 0.871       \\ \hline
				& 0.832   & 0.887  & 0.849       & 0.833  & 0.886 & 0.849      & 0.832   & 0.886  & 0.849       & 0.823  & 0.884  & 0.862       \\ \hline
			\end{tabular}
		}
		\caption{Kết quả train model dựa trên một số độ đo}
		\label{tab:table_4_8}
	\end{table}%
	\subsection{Kết quả train model phân loại loại tin rác}
	\begin{table}[H]
		\centering
		\setlength\extrarowheight{3pt}
		\begin{tabular}{|l|l|l|}
			\hline
			Dữ liệu gán nhãn & Số lượng \\
			\hline
			Mẫu dữ liệu   & 8281\\
			\hline
			Số chiều   & 9526\\
			\hline
			Cross-validation:   & 10\\
			\hline
		\end{tabular}%
		\caption{Thông số cơ bản của tập train} \label{tab:table_4_9}%
	\end{table}
	\begin{table}[H]
		\centering
		\begin{tabular}{|l|l|l|}
			\hline
			Time build model(seconds): &            &          \\
			\hline
			SVM                        & NaiveBayes & J48      \\
			\hline
			14.46                     & 19.39     & 590.78	\\
			\hline
		\end{tabular}
		\caption{Thời gian train model}
		\label{tab:table_4_10}
	\end{table}
	\begin{table}[H]
		\centering
		\rotatebox{90}{
			\begin{tabular}{|c|c|c|c|c|c|c|c|c|c|c|c|c|}
				\hline
				\multirow{2}{*}{Class} & \multicolumn{3}{c|}{Precision} & \multicolumn{3}{c|}{Recall} & \multicolumn{3}{c|}{F-Measure} & \multicolumn{3}{c|}{ROC Area} \\ \cline{2-13} 
				& NaiveBayes   & J48    & SVM    & NaiveBayes  & J48   & SVM   & NaiveBayes   & J48    & SVM    & NaiveBayes  & J48    & SVM    \\ \hline
				Chia sẻ                & 0.97         & 0.952  & 0.96   & 0.730       & 0.985 & 0.986 & 0.833        & 0.968  & 0.973  & 0.973       & 0.978  & 0.980  \\ \hline
				Quảng cáo              & 0.931        & 0.990  & 0.99   & 0.945       & 0.968 & 0.975 & 0.938        & 0.979  & 0.982  & 0.962       & 0.980  & 0.981  \\ \hline
				Tuyển dụng             & 0.335        & 0.989  & 1      & 0.992       & 0.952 & 0.955 & 0.5          & 0.970  & 0.977  & 0.954       & 0.987  & 0.977  \\ \hline
				& 0.919        & 0.975  & 0.979  & 0.862       & 0.974 & 0.979 & 0.877        & 0.974  & 0.979  & 0.966       & 0.980  & 0.981  \\ \hline
			\end{tabular}
		}
		\caption{Kết quả train model dựa trên một số độ đo}
		\label{tab:table_4_11}
	\end{table}%
\section{Nhận xét}
	\subsection{Nhận định về các thuật toán phân lớp cho bài toán}
	Dựa vào bảng \ref{tab:table_4_5}, \ref{tab:table_4_8} và \ref{tab:table_4_11}, ta thấy rằng với độ đo và tập dữ liệu mẫu, thuật toán Support Vector Machine cho kết quả tốt với mọi độ đo so với thuật toán Naive Bayes và J48.\\
	Dựa vào bảng \ref{tab:table_4_4}, \ref{tab:table_4_7} và \ref{tab:table_4_10}, ta thấy rằng thời gian để train model cho thuật toán Support Vector Machine tốn ít thời gian hơn thuật toán Naive Bayes và J48.
\section{Kết chương}
Qua các kết quả thử nghiệm, chương này đã thể hiện được một số tính chất của các thuật toán được đánh giá như thời gian train, một số độ đo thường dùng để đánh giá một hệ phân lớp. Ta thấy thuật toán SVM có thời gian train nhanh trong khi vẫn giữ được độ chính xác tốt so với thuật toán Naive Bayes và J48. Riêng thuật toán J48 có hạn chế về thời gian train quá lớn so với hai thuật toán còn lại.
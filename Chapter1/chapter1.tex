\chapter{MỞ ĐẦU}
\ifpdf
    \graphicspath{{Chapter1/Chapter1Figs/PNG/}{Chapter1/Chapter1Figs/PDF/}{Chapter1/Chapter1Figs/}}
\else
    \graphicspath{{Chapter1/Chapter1Figs/EPS/}{Chapter1/Chapter1Figs/}}
\fi

\section{Dẫn nhập}
Trong thời đại bùng bổ về thông tin, báo chí điện tử là một trong những nguồn thông tin lớn và quan trọng cần được khai thác và phân tích một cách hiệu quả. Hệ thống phát hiện tin nóng hiện đang được triển khai và hoạt động tại VCCorp là một trong các hệ thống phân tích và xử lý dữ liệu tin tức, được phát triển nhằm mục đích xác định tin nóng từ các trang báo điện tử Việt Nam. Tuy nhiên, sự đa dạng và phong phú của nguồn tin dẫn đến việc bản chất của nhiều bài thu thập được không phù hợp với mục tiêu của hệ thống. Những bài đó, đối với hệ thống, được hiểu là "rác". Số lượng "rác" lớn sẽ gây nhiễu và làm cho việc phân tích và xử lý trở nên kém chính xác. Một cơ chế lọc rác hiệu quả sẽ giúp cải thiện độ chính xác cũng như giảm tải  cho hệ thống từ việc phân tích và xử lý các dữ liệu không có giá trị sử dụng.

Bài toán đặt ra là làm thể nào để có thể lọc "rác" trong quá trình thu thập tin tức một các chính xác. Với nhu cầu và điều kiện thuận lời từ công ty VCCorp, khóa luận này hướng đến việc xây dựng một hệ thống có khả năng lọc "rác" để tăng hiệu quả cho hệ thống phát hiện tin nóng và giúp đánh giá độ đáng tin cậy của các nguồn tin.

\section{Mục tiêu đề tài}
	\begin{itemize}
		\item Tìm hiểu và đánh giá các thuật toán phân lớp cho việc lọc rác dữ liệu tin tức
		\item Xây dựng hệ thống lọc rác áp dụng các thuật toán phân lớp.
	\end{itemize}

\section{Nội dung thực hiện}
	\begin{itemize}
		\item Tìm hiểu bài toán phân loại tin tức, tìm hiểu các phương pháp và các hướng tiếp cận
		\item Thử nghiệm đánh giá các phương pháp đã tìm hiểu:
		\begin{itemize}
			\item Thu thập dữ liệu tin tức từ cơ sở dữ liệu của công ty VCCorp.
			\item Tiến hành một số thống kê trên dữ liệu thu thập được.
			\item Huấn luyện và so sánh các thuật toán phân lớp: Naive Bayes, SVM, J48.
		\end{itemize}
	
		\item Xây dựng hệ thống:
		\begin{itemize}
			\item Tìm hiểu về MongoDB, framework Struts 2, thư viện Apache Lucene, Weka, LibSVM.
			\item Xây dựng kiến trúc hệ thống.
			\item Thiết kế chức năng, giao diện hệ thống.
			\item Cài đặt hệ thống.
		\end{itemize}
		
	\end{itemize}

\section{Phạm vi đề tài}
\begin{itemize}
	\item Nguồn dữ liệu: các bài viết từ báo chính thống Việt Nam.
	\item Ngôn ngữ: tiếng Việt.
	\item Các thuật toán tìm hiểu: Naive Bayes, SVM, J48
\end{itemize}

\section{Cấu trúc báo cáo}
Luận văn được bố cục thành chương mục như sau:
\begin{itemize}
	\item \textbf{Chương 1}: Mở đầu: Giới thiệu về đề tài.
	\item \textbf{Chương 2}: Bài toán lọc rác và các phương pháp tiếp cận phổ biến: Trình bày cơ sở lý thuyết, các khái niệm, phương pháp tiếp cận liên quan đến bài toán lọc rác.
	\item \textbf{Chương 3}: Hiện thực hệ thống lọc rác: Trình bày về kiến trúc, cài đặt hệ thống phát hiện tin nóng.
	\item \textbf{Chương 4}: Thực nghiệm và đánh giá: Trình bày về bộ dữ liệu thu thập được, đánh giá và so sánh các thuật toán.
	\item \textbf{Mục Tài liệu tham khảo}
	\item \textbf{Phụ lục. Giới thiệu về thư viện Apache Lucene}
\end{itemize}

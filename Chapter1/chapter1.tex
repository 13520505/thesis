\chapter{MỞ ĐẦU}
\ifpdf
    \graphicspath{{Chapter1/Chapter1Figs/PNG/}{Chapter1/Chapter1Figs/PDF/}{Chapter1/Chapter1Figs/}}
\else
    \graphicspath{{Chapter1/Chapter1Figs/EPS/}{Chapter1/Chapter1Figs/}}
\fi

\section{Dẫn nhập}
Việc đọc báo cập nhật tin tức là nhu cầu thiết yếu của nhiều người. Ngày nay, với sự phát triển mạnh mẽ của các dịch vụ chia sẻ thông tin trên internet, ta có rất nhiều nguồn tin tức dồi dào để lựa chọn. Tuy nhiên, điều này cũng dẫn đến tình trạng quá tải thông tin, và việc tìm ra những "tin nóng" càng trở nên khó khăn. Tin nóng có thể hiểu là những tin có nội dung quan trọng, đáng chú ý, thu hút sự quan tâm của cộng đồng, và có thể gây ảnh hưởng rộng rãi. 

%Tin tức có thể đến từ nhiều nguồn như báo giấy, báo mạng, và các kênh truyền hình chính thống. Ngoài ra, các trang mạng xã hội cũng có thể đóng vai trò là một nguồn tin, với các tin tức được chia sẻ bởi chính những người dùng. Một số thống kê cho thấy: trong thời điểm đầu năm 2017, trên mạng xã hội Facebook cứ mỗi phút có khoảng 293,000 bài viết và 510,000 lượt bình luận được đăng lên\footnote{https://zephoria.com/top-15-valuable-facebook-statistics/}; trong khi đó số bài viết mỗi phút trên Twitter là 350,000\footnote{http://www.internetlivestats.com/twitter-statistics/}. Với những con số trên, ta có thể thấy mạng xã hội có tiềm năng trở thành một nơi cung cấp tin tức rất nhanh chóng, tuy rằng tính xác thực nội dung có thể không bằng các kênh tin truyền thống. Các biên tập viên báo chí có thể theo dõi và khai thác nguồn tin này để đưa tin tức đến cho người đọc một cách chính xác và sớm nhất.

Ngoài các nguồn tin tức truyền thống như báo giấy, báo mạng, và các kênh truyền hình, các trang mạng xã hội cũng có thể đóng vai trò là một nguồn tin tức lớn, với các tin tức được chia sẻ bởi chính những người dùng. Twitter là một mạng xã hội hướng tới việc chia sẻ thông tin một cách nhanh chóng, ngắn gọn, và có lượng người dùng toàn cầu hàng tháng lên đến 328 triệu \footnote{https://about.twitter.com/company}. Thống kê cho thấy: trong thời điểm đầu năm 2017, trên Twitter cứ mỗi phút có khoảng 350,000 tweet được đăng tải\footnote{http://www.internetlivestats.com/twitter-statistics/}. Với những con số trên, ta có thể thấy Twitter cũng như những mạng xã hội khác có tiềm năng trở thành một nơi cung cấp tin tức rất nhanh chóng, tuy rằng tính xác thực nội dung có thể không bằng các kênh tin truyền thống. 

Bài toán đặt ra là làm thể nào để có thể khai thác nguồn tin từ mạng xã hội, nhằm đưa tin tức đến cho người đọc một cách chính xác và sớm nhất. Với nhu cầu và điều kiện thuận lời từ công ty VCCorp, khóa luận này hướng đến việc xây dựng một hệ thống có khả năng lọc và phát hiện tin nóng để hỗ trợ cho các biên tập viên báo chí trong quá trình tìm kiếm tư liệu để viết bài.

%Ngày nay, với sự phát triển mạnh mẽ của các dịch vụ chia sẻ thông tin trên internet, đặc biệt là sự phổ biến của các trang mạng xã hội cùng với lượng bài viết hằng ngày khổng lồ, việc tìm ra những bài mang nội dung thú vị, hữu ích càng trở nên khó khăn.
%
%Một số thống kê cho thấy: trong thời điểm đầu năm 2017, trên mạng xã hội Facebook cứ mỗi phút có 293,000 bài viết, 510,000 lượt bình luận và hơn 136,000 tấm ảnh được đăng lên \footnote{https://zephoria.com/top-15-valuable-facebook-statistics/}; trong khi đó số bài viết mỗi phút trên Twitter là 350,000 \footnote{http://www.internetlivestats.com/twitter-statistics/}. Với những con số trên, ta có thể thấy song song với Facebook, Twitter đang dần khẳng định vị thế là một trong những trang mạng thông tin hiệu quả nhất. Nhưng đi kèm với đó cũng là những bài toán được đặt ra cho các nhà khai thác kênh tin tức, sự kiện này, làm thế nào để giúp người dùng tiếp cận với thông tin họ mong muốn một cách nhanh nhất, hiệu quả nhất, tránh những thông tin không liên quan.
%
%Nhìn chung, việc xây dựng một hệ thống có khả năng lọc và phát hiện những bài viết, tin tức tiềm năng hoặc có tầm ảnh hưởng rộng có thể giúp cho các biên tập viên báo chí, các nhà phân tích cũng như người dùng sớm nắm bắt được thông tin về những sự kiện đang diễn ra xung quanh họ.
%
%Hơn thế nữa, với bất cứ một người sử dụng Twitter nào, việc tiếp cận ngay với nguồn tin tức mong muốn tiết kiệm không ít thời gian sử dụng mạng xã hội của họ nói chung. Thực tế cho thấy, không ít người dùng quan tâm đến một lĩnh vực cụ thể nhưng liên tục bị làm phiền bởi các tin tức không liên quan xuất hiện xung quanh trên Twitter.
%
%Nhận thức được những vấn đề này, cùng với cơ hội đang được thực tập trong lĩnh vực liên quan, em đã chọn đề tài: "Xây dựng hệ thống phát hiện sớm tin nóng từ mạng xã hội Twitter" làm khóa luận tốt nghiệp. 

%https://zephoria.com/top-15-valuable-facebook-statistics/
%http://www.internetlivestats.com/twitter-statistics/#sources
\section{Mục tiêu đề tài}
	\begin{itemize}
%		\item Tìm hiểu một số phương pháp để tìm ra được bài viết mới nhất về sự kiện nào đó.
%		\item Thu thập bộ dữ liệu các bài đăng trên tweet.
%		\item So sánh các thuật toán phát hiện tin nóng trên tập dữ liệu thu thập được.
		\item Tìm và chọn được phương pháp phù hợp nhất để nhận biết tin nóng, là tin về những sự kiện mới, có khả năng thu hút sự chú ý, quan tâm của nhiều người.
		\item Xây dựng hệ thống áp dụng phương pháp trên để hỗ trợ biên tập viên trong việc viết bài.
	\end{itemize}

\section{Nội dung thực hiện}
%	\begin{itemize}
%		\item Tìm hiểu lý thuyết:
%		\begin{itemize}
%			\item Đọc và thảo luận các paper liên quan đến bài toán phát hiện tin tức.
%			\item Tìm hiểu và trình bày về MongoDB.
%			\item Tìm hiểu cách sử dụng Twitter API.
%			\item Tìm hiểu framework Struts 2.
%			\item Tìm hiểu thư viện Apache Lucene.
%		\end{itemize}
%	
%		\item Thực hiện:
%		\begin{itemize}
%			\item Thiết kế chức năng, giao diện hệ thống.
%			\item Xây dựng kiến trúc hệ thống.
%			\item Thu thập dữ liệu Twitter thông qua Twitter Search API và Streaming API.
%			\item Thiết lập sử dụng MongoDB để lưu trữ dữ liệu cho hệ thống.
%			\item Tiến hành một số thống kê trên dữ liệu thu thập được.
%			\item Cài đặt module tiền xử lý dữ liệu và các thuật toán gom cụm.
%			\item Thử nghiệm so sánh hiệu năng các thuật toán gom cụm.
%		\end{itemize}
%	\end{itemize}

	\begin{itemize}
		\item Tìm hiểu bài toán phát hiện tin nóng: Đọc và thảo luận các bài báo nghiên cứu về phát hiện tin tức trên mạng xã hội.
		
		\item Thử nghiệm đánh giá các phương pháp đã tìm hiểu:
		\begin{itemize}
			\item Thu thập dữ liệu Twitter thông qua Twitter Search API và Streaming API.
			\item Tiến hành một số thống kê trên dữ liệu thu thập được.
			\item Cài đặt và so sánh các thuật toán: k-láng giềng gần nhất, gom cụm có tăng trọng số cho thực thể có tên, và gom cụm dùng Locality Sensitive Hashing.
		\end{itemize}
	
		\item Xây dựng hệ thống:
		\begin{itemize}
			\item Tìm hiểu về MongoDB, framework Struts 2, thư viện Apache Lucene.
			\item Xây dựng kiến trúc hệ thống.
			\item Thiết kế chức năng, giao diện hệ thống.
			\item Cài đặt hệ thống.
		\end{itemize}
		
	\end{itemize}

\section{Phạm vi đề tài}
\begin{itemize}
	\item Nguồn dữ liệu: các bài viết từ mạng xã hội Twitter.
	\item Ngôn ngữ: tiếng Việt.
	\item Các phương pháp tiếp cận: k-láng giềng gần nhất, gom cụm có tăng trọng số cho thực thể có tên, và gom cụm dùng Locality Sensitive Hashing. Kết hợp với xếp hạng cụm.
\end{itemize}

%\section{Kết quả đạt được}

\section{Cấu trúc báo cáo}
Luận văn được bố cục thành chương mục như sau:
\begin{itemize}
	\item \textbf{Chương 1}: Mở đầu: Giới thiệu về đề tài.
	\item \textbf{Chương 2}: Bài toán phát hiện tin nóng và các phương pháp tiếp cận phổ biến: Trình bày cơ sở lý thuyết, các khái niệm, phương pháp tiếp cận liên quan đến bài toán phát hiện tin nóng.
	\item \textbf{Chương 3}: Hiện thực hệ thống phát hiện tin nóng: Trình bày về kiến trúc, cài đặt hệ thống phát hiện tin nóng.
	\item \textbf{Chương 4}: Thực nghiệm và đánh giá: Trình bày về bộ dữ liệu thu thập được, đánh giá và so sánh các thuật toán.
	\item \textbf{Mục Tài liệu tham khảo}
	\item \textbf{Phụ lục. Giới thiệu về thư viện Apache Lucene}
\end{itemize}

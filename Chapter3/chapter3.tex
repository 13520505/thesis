\chapter{HIỆN THỰC HỆ THỐNG LỌC RÁC TIN TỨC}
\ifpdf
    \graphicspath{{Chapter3/Chapter3Figs/PNG/}{Chapter3/Chapter3Figs/PDF/}{Chapter3/Chapter3Figs/}}
\else
    \graphicspath{{Chapter3/Chapter3Figs/EPS/}{Chapter3/Chapter3Figs/}}
\fi

\section{Mở đầu}
Chương này sẽ trình bày về thiết kế, chi tiết cài đặt, các thư viện và framework được sử dụng để xây dựng hệ thống lọc rác tin tức.

\section{Mô hình hệ thống}
Dưới đây là mô hình của các thành phần chính hệ thống:
\begin{figure}[H]
	\centering
	\includegraphics[width=0.9\linewidth]{Chapter3/Chapter3Figs/PDF/SystemArchitecture}
	\caption{Các thành phần chính của hệ thống}
	\label{fig:systemarchitecture}
\end{figure}

Hệ thống chủ yếu phục vụ cho biên tập viên. Biên tập viên sử dụng hệ thống thông qua giao diện Spam Filtering để sử dụng chức năng phân lớp tin tức hoặc xem các số liệu thống kê về tin tức đã phân lớp lưu trong cơ sở dữ liệu.
%\begin{itemize}
%	\item \textit{Data Streaming}: Module này sử dụng Twitter Streaming API để thu thập các bài đăng từ Twitter, theo một danh sách các từ khóa cho trước.
%	\item \textit{Data Preprocessor}: Module tiền xử lý (loại bỏ link, stopwords; tách từ) và biểu diễn dữ liệu thành vector trọng số tf-idf.
%	\item \textit{Clustering Algorithm}: Chạy thuật toán gom cụm trên dữ liệu đã mã hóa thành vector trọng số, đưa ra các cụm bài viết tương ứng và lưu vào cơ sở dữ liệu.
%	\item \textit{MongoDB}: Cơ sở dữ liệu chứa bài viết và các thông tin đính kèm, sử dụng MongoDB.
%	\item \textit{Top Stories}: Giao diện hiển thị các cụm tin tức đáng chú ý cho người dùng.
%\end{itemize}

\section{Phân hệ thu thập dữ liệu (Data Streaming)}
Module này sử dụng hệ thống Crawler để lấy các bài đăng từ nhiều nguồn tin tức lưu trữ ở cơ sở dữ liệu MySQL và đưa qua cơ sở dữ liệu MongoDB, hệ thống sẽ thực hiện các chức năng:
	\begin{itemize}
		\item Connect vào cơ sở dữ liệu mySQL để lấy các bài bài báo.
		\item Xóa các bài viết trùng.
		\item Lưu trữ xuống cơ sở dữ liệu MongoDB
	\end{itemize}
\section{Phân hệ tiền xử lý dữ liệu (Data Preprocessor)}
Module tiền xử lý có nhiệm vụ chính gồm loại bỏ URL, thực hiện tách từ, loại bỏ stopwords và biểu diễn dữ liệu thành vector trọng số tf-idf. Trước khi chạy thuật toán phân loại, dữ liệu được lấy ra từ MongoDB và tiền xử lý phần nội dung của bài báo bằng các bước sau:
	\begin{enumerate}
		\item Loại bỏ URL bằng regular expression.
		\item Tách từ sử dụng thư viện TPSegmenter. Bước này dùng để biểu diễn các từ ghép trong tiếng Việt bằng cách thêm gạch nối giữa các tiếng của từ.\\
		Ví dụ: \textit{"Vụ tai nạn 13 người chết: Đã giám định mẫu máu tài xế xe tải"} qua bộ tách từ sẽ thành \textit{"Vụ tai\_nạn 13 người chết : Đã giám\_định mẫu máu tài\_xế xe\_tải"}.
		\item Loại bỏ các từ trong danh sách gồm 813 stopwords.
		\item Tính và biểu diễn dữ liệu thành vector tf-idf và metadata.
	\end{enumerate}

\section{Phân hệ phân loại tin tức}
Sử dụng model đã train trước đó để phân loại tin tức. Model sử dụng thuật toán SVM đã trình bày ở chương 2 làm thuật toán chính để phân loại tin tức. Hiện tại hệ thống có 2 nhãn đã được định nghĩa trước đó:
	\begin{itemize}
		\item \textit Không rác
		\item \textit Rác
	\end{itemize}

Sau khi đã phân loại, hệ thống sẽ lưu nhãn đã gán cho tin tức xuống cơ sở dữ liệu MongoDB.

\section{Phân hệ phân loại loại rác}
Hệ thống sẽ lấy những tin được gán nhãn "Rác" ở bước phân loại tin tức ở trên để phân loại loại rác cho tin đó. Hiện tại có 3 loại rác đã được định nghĩa trước:
\begin{itemize}
	\item \textit Quảng cáo
	\item \textit Tuyển dụng
	\item \textit Chia sẻ
\end{itemize}

\section{Thiết kế hệ thống} 
	\begin{figure}[H]
		\centering
		\includegraphics[width=0.5\linewidth]{Chapter3/Chapter3Figs/CrawlerSystem}
		\caption{Kiến trúc hệ thống crawler tin tức}
		\label{fig:crawlersystem}
	\end{figure}
Hệ thống Crawler được xây dựng độc lập bao gồm phân hệ thu thập dữ liệu, phân hệ tiền xử lý và phân hệ phân lớp để chạy ngầm nhằm thu thập và phân loại thông tin
	\begin{figure}[H]
		\centering
		\includegraphics[width=0.5\linewidth]{Chapter3/Chapter3Figs/Layers}
		\caption{Kiến trúc hệ thống kết hợp Multilayer architecture kết hợp với mô hình MVC}
		\label{fig:layers}
	\end{figure}
%Hệ thống xây dựng theo kiến trúc client-server và sử dụng struts 2 framework áp dụng mô hình Model - View - Controller (MVC) với View tách biệt(sử dụng framework React) và cung cấp api để trả về kết quả dạng JSON. Cụ thể:
Hệ thống xây dựng theo kiến trúc 3 tầng gồm: Presentation Layer, Business Logic Layer và Data Access Layer. Trong đó, Presentation Layer áp dụng mô hình Model - View - Controller (MVC). Cụ thể:
	\begin{itemize}
		\item Presentation Layer: Có trách nhiệm hiển thị thông tin, tương tác với người dùng hệ thống. Gồm 2 thành phần:
			\begin{itemize}
				\item Controller: Điều khiển các luồng của hệ thống web, nhận các tín hiệu từ người dùng và xử lý tương ứng.
				\item View: Có nhiệm vụ hiển thị các giao diện hệ thống cho người dùng, hệ thống sử dụng React để gọi api trả về dữ liệu cho giao diện người dùng.
			\end{itemize}
		\item Model: Đối tượng chứa dữ liệu để xử lý và hiển thị.
		\item Business Layer: Chứa các nghiệp vụ của hệ thống. Bao gồm các bước xử lý dữ liệu, thuật toán gom cụm, các tác vụ thống kê,…
		\item Data Access Layer: Có nhiệm vụ giao tiếp với các hệ cơ sở dữ liệu.
	\end{itemize}

\section{Cài đặt hệ thống}%các packages
Hệ thống ứng dụng được xây dựng trên nền tảng Java EE với các thành phần sau:
	\begin{itemize}
		\item Ngôn ngữ: Java, HTML, CSS, JavaScript, React.
		\item Hệ cơ sở dữ liệu: MongoDB và MySQL.
		\item Thư viện, framework: Apache Struts 2, Apache Lucene, TPSegmenter, Weka
		\item Server: Apache Tomcat
	\end{itemize}

	\subsection{Các package}
	Source code chương trình được tổ chức thành các package như sau:\\
	Hệ thống crawler:
	\begin{itemize}
		\item vn.vccorp.crawler.bo: Chứa các business object của hệ thống
		\item vn.vccorp.crawler.config: Các file config cho hệ thống
		\item vn.vccorp.crawler.constant: Các file constant của hệ thống
		\item vn.vccorp.crawler.dao: Data Access Object, thực hiện các tác vụ đọc, ghi database
		\item vn.vccorp.crawler.dbconnection: Cung cấp kết nối đến database
		\item vn.vccorp.crawler.dto: Data Transfer Object, các đối tượng để vận chuyển dữ liệu từ database
		\item vn.vccorp.crawler.main: Các lớp bao đóng để tạo thread chạy song song các tác vụ
		\item vn.vccorp.crawler.thread: Các lớp bao đóng để tạo thread chạy song song các tác vụ
		\item vn.vccorp.crawler.util: Các công cụ hỗ trợ trong hệ thống
	\end{itemize}
	Hệ thống HotNewsDetector:
	\begin{itemize}
		\item vn.vccorp.hotnewsdetector.action.general: lớp ảo chứa các phương thức của action.
		\item vn.vccorp.hotnewsdetector.action.news: chứa các action cho tin tức
		\item vn.vccorp.hotnewsdetector.action.twitter: chứa các action cho twitter
		\item vn.vccorp.hotnewsdetector.bo: Chứa các business object của hệ thống
		\item vn.vccorp.hotnewsdetector.config: Các file config cho hệ thống
		\item vn.vccorp.hotnewsdetector.constant: Các file constant của hệ thống
		\item vn.vccorp.hotnewsdetector.context: chứa các action context của hệ thống
		\item vn.vccorp.hotnewsdetector.crawler.news: Dùng để lấy dữ liệu từ trang web tin tức
		\item vn.vccorp.hotnewsdetector.dao: Data Access Object, thực hiện các tác vụ đọc, ghi database
		\item vn.vccorp.hotnewsdetector.dbconnection: Cung cấp kết nối đến database
		\item vn.vccorp.hotnewsdetector.dto: Data Transfer Object, các đối tượng để vận chuyển dữ liệu từ database
		\item vn.vccorp.hotnewsdetector.exception: Quản lý lỗi và đưa ra thông báo của hệ thống
		%\item vn.vccorp.hotnewsdetector.lucence: 
		%\item vn.vccorp.hotnewsdetector.luceneindex:
		\item vn.vccorp.hotnewsdetector.thread
		\item vn.vccorp.hotnewsdetector.utils
	\end{itemize}
	
	\subsection{Cơ sở dữ liệu MongoDB}
	Hệ thống sử dụng MongoDB để lưu trữ dữ liệu tin tức và quản lý kết quả phân lớp. Đây là một hệ cơ sở dữ liệu NoSQL, cung cấp khả năng mở rộng, sao lưu, phân mảnh dữ liệu tốt, và có thể thay đổi cấu trúc dữ liệu một cách linh hoạt.
	
	Dưới đây là bảng so sánh một số thuật ngữ cơ bản giữa các cơ sở dữ liệu SQL truyền thống và MongoDB:
	\begin{table}[H]
		\centering
		\setlength\extrarowheight{3pt}
		\begin{tabular}{|l|l|}
			\hline
			\textbf{Thuật ngữ SQL}	& \textbf{Thuật ngữ MongoDB}  \\\hline
			database	& database \\\hline
			table		& collection \\\hline
			row			& document hoặc BSON document \\\hline
			column		& field \\\hline
			index		& index \\\hline
			table joins	& \$lookup, embedded documents \\\hline
		\end{tabular}
		\caption{So sánh các thuật ngữ giữa SQL và MongoDB}
		\label{tab:table_3_1}
	\end{table}
	
		\subsubsection{Collection News}
		Collection này chứa thông tin về các tin lấy từ cơ sở dữ liệu MySQL, cũng là nơi lưu trữ tất cả thông tin về tweet trong hệ thống. Khi dữ liệu được stream từ MySQL về, mỗi tin chỉ có 12 trường, các trường khác sẽ được thêm vào trong quá trình hệ thống xử lý.
		\begin{table}[H]
			%				\centering
			\setlength\extrarowheight{3pt}
			\begin{tabular}{|l|l|p{7.25cm}|}
				\hline
				\textbf{Thuộc tính}     & \textbf{Loại} & \textbf{Ý nghĩa} \\\hline
				\_id           & ObjectID       & ID trong MongoDB của document \\\hline
				Id        & Int           & ID của tin tức lấy về\\\hline
				Title         & String           & Title của tin tức\\\hline
				Content & String         & Nội dung của tin tức\\\hline
				Source   & String         & Nguồn trang báo điện tử của tin tức\\\hline
				CreateTime  & Date           & Thời gian đăng của tin\\\hline
				GetTime    & Date           & Thời gian tin tức được thu thập vào cơ sở dữ liệu MySQL\\\hline
				CollectDate    & Date           & Thời gian tweet được thu thập vào cơ sở dữ liệu MongoDB\\\hline
				Author  & String        & Người viết bài báo(nếu có)\\\hline
				Category   & Integer        & Chủ đề của bài báo\\\hline
				SpamLabel & String        & Nhãn đã gán cho tin tức\\\hline
				SpamCategory   & String         & Nhãn đã gán cho loại tin rác\\\hline
				SpamLabelFeedback      & Integer        & Feedback của nhãn đã gán cho tin tức đó\\\hline
			\end{tabular}%
			
			\caption{Các trường của collection News}
			\label{tab:table_3_2}%
		\end{table}%
		
\section{Giao diện}
Giao diện hệ thống được xây dựng sử dụng thư viện React, với kiến trúc Redux. Giao diện tương tác với hệ thống thông qua ajax request.
	\begin{figure}[H]
		\centering
		\includegraphics[width=1\linewidth]{Chapter3/Chapter3Figs/FrontEndArch}
		\caption{Kiến trúc giao diện hệ thống}
		\label{fig:layers}
	\end{figure}

  \subsection{Giao diện danh sách tin đã phân lớp}
  \begin{table}[H]
    \centering
    \setlength{\tabcolstep}{12pt}
    \begin{tabular}{@{}lll@{}} \toprule
      Tên  & Loại   & Mô tả \\ \midrule
      fetchSpam &  Action   & Truy xuất danh sách các tin đã phân lớp \\
      sendFeedback & Action & Gửi phản hồi người dùng về label của tin đã phân lớp \\
      spamList     & State       & Danh sách các tin đã phân lớp  \\ \bottomrule
    \end{tabular}
    \caption{Bảng các thuộc tính}
  \end{table}

  \begin{table}[H]
    \centering
    \setlength{\tabcolstep}{12pt}
    \begin{tabular}{@{}lll@{}} \toprule
      Tên  & Loại   & Mô tả \\ \midrule
      Date  &  Date Picker & Chọn ngày để lấy danh sách tin đã phân lớp \\
      SpamTable  & Table & Hiển thị danh sách các tin đã phân lớp \\
      FeedbackBox & Select & Chọn và gửi phản hồi về label của tin đã phân lớp \\ \bottomrule
    \end{tabular}
    \caption{Bảng các đối tượng hiển thị}
  \end{table}

  \subsection{Giao diện phân lớp tin}
  \begin{table}[H]
    \centering
    \setlength{\tabcolstep}{12pt}
    \begin{tabular}{@{}lll@{}} \toprule
      Tên  & Loại   & Mô tả \\ \midrule
      fetchLabels &  Action   &  Gửi danh sách các tin cần phân lớp và lấy về các label tương ứng \\
      spamLabels  &  State  & Danh sách các tin và label tương ứng \\ \bottomrule
    \end{tabular}
    \caption{Bảng các thuộc tính}
  \end{table}

  \begin{table}[H]
    \centering
    \setlength{\tabcolstep}{12pt}
    \begin{tabular}{@{}lll@{}} \toprule
      Tên  & Loại   & Mô tả \\ \midrule
      InputTable   &  Table & Bảng chứa các tin cần hoặc đã phân lớp \\
      DatatypeMenu   & Tabs & Chuyển kiễu dữ liệu của input: text hoặc URL \\
      ClassifyButton  & Button  & Thực hiện phân lớn các input \\
      InputButton   & Button & Hiển thị input form \\
      InputForm   & Form, Popup & Form để thêm input mới \\
      ContentText & Input & Nội dung input \\
      ExpectedLabel & Select & Label mong đợi của input \\
      ConfirmButton & Button & Thêm input mới vào table của tab hiện tại \\
      ImportButton & Button & Truyền dữ liệu input từ file \\
      ClearAllButton & Button & Xóa hết dữ liệu input từ table của tab hiện tại \\
      DeleteButton & Button & Xóa một dòng input trên bảng \\ \bottomrule
    \end{tabular}
    \caption{Bảng các đối tượng hiển thị}
  \end{table}
  \subsection{Giao diện thống kê dữ liệu phân lớp}
  \begin{table}[H]
    \centering
    \setlength{\tabcolstep}{12pt}
    \begin{tabular}{@{}lll@{}} \toprule
      Tên  & Loại   & Mô tả \\ \midrule
      fetchSpamStatistics &  Action   &  Truy xuất dữ liệu thống kê phân lớp \\
      spamStatistics  &  State  & Dữ liệu thống kê phân lớp \\ \bottomrule
    \end{tabular}
    \caption{Bảng các thuộc tính}
  \end{table}

  \begin{table}[H]
    \centering
    \setlength{\tabcolstep}{12pt}
    \begin{tabular}{@{}lll@{}} \toprule
      Tên  & Loại   & Mô tả \\ \midrule
      DateRange   &  Select & Chọn khung thời gian để lấy dữ liệu thống kê \\
      ChartsSlider   & Slider &  Danh sách các đồ thị hiển thị dữ liệu thống kê \\ \bottomrule
    \end{tabular}
    \caption{Bảng các đối tượng hiển thị}
  \end{table}
\section{Kết quả}
Hệ thống đã hoàn thiện các chức năng: lọc rác tin tức trên Crawler, hiển thị danh sách tin đã qua phân lớp bằng bộ lọc rác, bộ phân lớp lọc rác thủ công, thống kê dữ liệu đã phân lớp. Dưới đây là một số hình ảnh của hệ thống:
\begin{figure}[H]
	\centering
	\includegraphics[width=1\linewidth]{Chapter3/Chapter3Figs/Classified.png}
	\caption{Giao diện danh sách tin đã phân lớp. Người dùng có thể chọn ngày để hiển thị tin và gửi phản hồi về nhãn của tin.}
	\label{fig:streaming}
\end{figure}

\begin{figure}[H]
	\centering
	\includegraphics[width=0.96\linewidth]{Chapter3/Chapter3Figs/Chart1.png}
  \caption{Giao diện thống kê biểu đồ với hai biểu đồ. Biểu đồ thứ nhất thể hiện tổng số tin đã phân lớp phân theo nhãn "Rác" và "Không rác". Biểu đồ thứ hai thể hiện số lượng và tỉ lệ số lượng tin rác của mỗi nguồn tin so với tổng số tin rác và số tin rác của các nguồn tin khác (biểu đồ chỉ hiển thị top 20, nguồn tin có số lượng tin rác nhiều nhất)}
	\label{fig:streamingkeywords}
\end{figure}

\begin{figure}[H]
	\centering
	\includegraphics[width=0.96\linewidth]{Chapter3/Chapter3Figs/Chart2.png}
	\caption{Giao diện thống kê. Biểu đồ thể hiện số lượng tin "Rác" và "Không rác" và tỉ lệ tin "Rác" so với tổng số tin theo thời gian. Người dùng có thể chọn hiển thị số liệu thống kê trong hôm nay, 7 ngày trước, 30 ngày trước, hoặc 12 tháng trước }
	\label{fig:streamingkeywords}
\end{figure}

\begin{figure}[H]
	\centering
  \includegraphics[width=0.96\linewidth]{Chapter3/Chapter3Figs/Chart3.png}
  \caption{Giao diện thống kê. Biểu đồ thể hiện tỉ lệ tin rác của một nguồn tin trên tổng số các tin của nguồn tin đó. Biểu đồ chỉ hiển thị top 20 nguồn tin với tỉ lệ lớn nhất và khác 0}
	\label{fig:streamingkeywords}
\end{figure}

\begin{figure}[H]
		\centering
	\includegraphics[width=0.96\linewidth]{Chapter3/Chapter3Figs/Filter.png}
	\caption{Giao diện bộ phân lớp tin tức. Người dùng có thể thực hiện phân lớp một cách thủ công bằng cách cho cho vào input là văn bản hoặc các đường dẫn đến các bài báo cần phân lớp, kèm theo là nhãn mà người dùng gán cho tin đó. Input có thể được truyền vào bằng cách nhập từ bàn phím hoặc truyền vào từ một file. Sao khi đã hoàn thành việc phân lớp, hệ thống sẽ trả về các nhãn tương ứng kèm theo độ chính xác của việc phân lớp và confusion matrix}
	\label{fig:startclustering}
\end{figure}

\section{Kết chương}
Chương này đã trình bày về các thành phần chính hệ thống, kiến trúc phân tầng, các hệ cơ sở dữ liệu được sử dụng và cách tổ chức, cùng một số kết quả cài đặt hệ thống.
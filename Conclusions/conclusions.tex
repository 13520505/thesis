\chapter*{\begin{center}KẾT LUẬN \\VÀ HƯỚNG PHÁT TRIỂN \end{center}}
\addcontentsline{toc}{chapter}{KẾT LUẬN VÀ HƯỚNG PHÁT TRIỂN}
\ifpdf
    \graphicspath{{Conclusions/ConclusionsFigs/PNG/}{Conclusions/ConclusionsFigs/PDF/}{Conclusions/ConclusionsFigs/}}
\else
    \graphicspath{{Conclusions/ConclusionsFigs/EPS/}{Conclusions/ConclusionsFigs/}}
\fi

\def\baselinestretch{1.66}

\section*{Kết quả đạt được}
%Viết lại chỗ này. Dan dat tu mục tiêu đề tài lúc đầu là gì? Cách tiếp cận và các nội dung cần thực hiện lúc đầu là gì? Chọn lựa phương pháp ra sao? Cái nào tốt? Tốt cái gì? Xấu cài gì? Tóm lại kết quả đạt được sau thời gian thực hiện đề tài lài gì? Phân loại (Kiến thức, Sản phẩm mềm: Bộ dữ liệu, Chương trình cài đặt các thuật toán ABC, Kết quả thực nghiệm đánh giá, so sánh các thuật toán ABC, hệ thống nhận biết tin nóng ... hiện điện trang khai sử dụng tại VCCorp), Liệt kê...


\addcontentsline{toc}{section}{Kết quả đạt được}
Mục tiêu chính của đề tài là xâu dụng một hệ thống lọc rác tin tức để cải thiện tính hiệu quả cửa hệ thống phát hiện tin nóng. Một số nội dung thực hiện được đề ra ban đầu là: tìm hiểu bài toán liên quan, thu thập dữ liệu, cài đặt và đánh giá một số thuật toán, và xây dựng hệ thống. 

Sau quá trình nghiên cứu và thực hiện, khóa luận đã thu được một số kết quả sau:
	\begin{itemize}
		\item Kiến thức:
		\begin{itemize}
			\item Tìm hiểu về bài toán phân lớp văn bản, phát hiện rác tin tức.
%			\item Tìm hiểu về các công ng
		\end{itemize}
		
		\item Sản phẩm:
		\begin{itemize}
			\item Thu thập bộ dữ liệu từ cơ sở dữ liệu của công ty VCCorp.
			\item Khảo sát và đánh giá các các thuật toán phân lớp.
			\item Xây dựng được hệ thống có khả năng phát hiện rác dữ liệu tin tức đang được triển khai tại công ty VCCorp.
		\end{itemize}
	\end{itemize}
%Với mục tiêu hỗ trợ biên tập viên nhanh chóng phát hiện tin nóng trên mạng xã hội Twitter, khóa luận đã đạt được một số kết quả nhất định như sau:
%	\begin{itemize}
%		\item Khảo sát và đánh giá các phương pháp phát hiện tin nóng: Nearest Neighbor, Boost Named Entity, Locality Sensitive Hashing.
%		\item Xây dựng được hệ thống có khả năng phát hiện sớm các tin tức đáng chú ý.
%	\end{itemize}

\section*{Hướng phát triển}
\addcontentsline{toc}{section}{Hướng phát triển}
Tuy hệ thống đạt được kết quả tương đối khá tốt trong việc phát hiện rác cho dữ liệu tin tức, độ chính xác vẫn còn có thể cải thiện thêm nữa bằng việc cải thiện mô hình thông qua phản hồi người dùng và sử dụng các phương pháp, thuật toán khác.

Trong tương lai, hệ thống sẽ được phát triển thêm chức năng tự động cập nhật mô hình dựa trên phản hồi người dùng, và cho phép người dùng quản lý các mô hình đã huấn luyện. Hệ thống cũng có thể mở rộng để lọc rác cho các nguồn tin khác như mạng xã hội.
%Cải thiện model thu thập ý kiến btv
%Đóng gói thuật toán
\pagestyle{fancy}
    \lhead{}    \chead{}         	\rhead{}
    \lfoot{}    \cfoot{\thepage}	\rfoot{}
    \renewcommand{\headrulewidth}{0.4pt}
    \renewcommand{\footrulewidth}{0.4pt}

\chapter*{\centering TÓM TẮT KHÓA LUẬN}
\addcontentsline{toc}{chapter}{TÓM TẮT KHÓA LUẬN}
 
\ifpdf
    \graphicspath{{Abstract/AbstractFigs/PNG/}{Abstract/AbstractFigs/PDF/}{Abstract/AbstractFigs/}}
\else
    \graphicspath{{Abstract/AbstractFigs/EPS/}{Abstract/AbstractFigs/}}
\fi

Cùng với sự phát triển nhanh chóng của internet là sự tăng trưởng mạnh mẽ của ngành truyền thông báo chí. Cụ thể, tính đến ngày 31/12/2015, cả nước Việt Nam có 105 cơ quan báo điện tử, 207 trang thông tin điện tử tổng hợp của các cơ quan báo chí \footnote{http://mic.gov.vn/Pages/TinTuc/116095/Tinh-hinh-phat-trien-linh-vuc-bao-chi-va-phat-thanh-truyen-hinh-nam-2015.html}, và cứ mỗi giờ có khoảng 2,118 bài được đăng trên các trang báo mạng Việt Nam \footnote{Theo thống kê từ dữ liệu thu thập bởi công ty VCCorp tính đến tháng 7/2017.}. Cùng với sự tăng trưởng lớn về thông tin là nhu cầu phân tích và xử lý các thông tin đó để phục vụ cho nhiều mục tích khác nhau, và ngày nay nhiều doanh nghiệp đã và đang xây dựng các hệ thống phân tích dữ liệu tin tức để phục vụ tốt hơn cho người dùng và biên tập viên. 

Việc phân tích dữ liệu tin tức đòi hỏi hệ thống phải thu thập dữ liệu từ nhiều website tin tức khác nhau để có được một nguồn dữ liệu đa dạng, phong phú và đủ lớn. Tuy nhiên, dữ liệu thu thập từ các website có bản chất không phù hợp với mục đích của hệ thống phân tích có thể gây ra nhiễu và khiến cho kết quả phân tích, xử lý không chính xác. Do đó, việc sàn lọc dữ liệu thu thập được trước khi đưa vào phân tích và xử lý là cần thiết. 

Hiểu được nhu cầu trên, chúng em thực hiện khóa luận này với mục tiêu xây dựng một hệ thống lọc "rác" cho dữ liệu tin tức từ các nguồn báo chính thống Việt Nam, cụ thể là lọc "rác" cho hệ thống "Phát hiện tin nóng". 

Trong phạm vi của khóa luận, chúng em đã nghiên cứu một số thuật toán phân lớp: SVM, Naive Bayes, J48.

Sau quá trình thực hiện, đề tài khóa luận thu thập được các kết quả như sau:
	\begin{itemize}
		\item Thu thập bộ dữ liệu gồm các bài đăng trên các website tin tức Việt Nam.
		\item Đánh giá và so sánh các thuật toán: SVM, Naive Bayes, J48.
		\item Xây dựng được hệ thống lọc rác cho hệ thống phát hiện tin nóng.
	\end{itemize}


%============================
 
%Đứng trước khối lượng thông tin khổng lồ và ngày càng gia tăng với tốc độ chóng mặt nhờ sự phát triển của công nghệ số, các biên tập viên, nhà báo gặp phải nhiều khó khăn trong việc làm sao lựa chọn được thông tin đúng, phù hợp với từng lĩnh vực. Điều này đòi hỏi những người đưa tin phải dành thêm rất nhiều công sức, nguồn lực cho việc tìm kiếm, sàng lọc các nguồn tin, cũng như rất nhiều thời gian để kiểm chứng độ tin cậy của các tin đó.

%Khóa luận này được thực hiện nhằm tìm hiểu các phương pháp máy học áp dụng trong bài toán phát hiện tin nóng, với phạm vi là các bài viết từ mạng xã hội Twitter. Đồng thời xây dựng một hệ thống có khả năng nhận biết các sự kiện đáng chú ý vừa xảy ra, và thông báo cho người dùng một cách sớm nhất có thể.